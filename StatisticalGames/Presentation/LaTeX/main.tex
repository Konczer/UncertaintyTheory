\documentclass{beamer}

\usepackage[utf8]{inputenc}
\usetheme{Berlin} 
%\usetheme{Darmstadt}
\useinnertheme{rounded}

\usefonttheme{serif}
\usecolortheme[RGB={0,10,100}]{structure} 
\setbeamertemplate{items}[circle] 
\setbeamertemplate{navigation symbols}{} 

\usepackage{amsmath, amsthm, amssymb}
\usepackage{amsfonts}

\usepackage{graphics}
\usepackage{graphicx}
\usepackage{hyperref}
\usepackage{multicol}


%\usepackage{textcomp}

\usepackage{float}

\setbeamertemplate{blocks}[rounded][shadow=false]

\begin{document}

\setbeamertemplate{background}{\includegraphics[]{img/Background.png}} 



\title[Statistical Games]{Statistical Games \\ {\small Playful approach to statistics}}
\author{József Konczer}
\date{3. May 2024}

\begin{frame}
\titlepage
\end{frame}




\begin{frame}{Overview}

\end{frame}

\section{Introduction}

\begin{frame}{}





\end{frame}


\begin{frame}{}

\end{frame}


\begin{frame}{}

\end{frame}

\section{Fisher games}
\subsection{Definition}
\begin{frame}{}

\end{frame}


\begin{frame}{}


\end{frame}




\begin{frame}{}

\begin{equation}
    A
\end{equation}

\pause

\begin{equation}
    B
\end{equation}

\pause

\begin{equation}
    C
\end{equation}

\end{frame}

\begin{frame}{}

\end{frame}

\begin{frame}{}

\end{frame}

\begin{frame}{}

\end{frame}

\begin{frame}{}

\end{frame}

\begin{frame}{}

\end{frame}

\begin{frame}{}

\end{frame}

\subsection{Equilibrium solution}

\begin{frame}{}

\end{frame}

\begin{frame}{}

\end{frame}

\begin{frame}{}

\end{frame}

\begin{frame}{}

\end{frame}

\begin{frame}{}

\end{frame}

\begin{frame}{}

\end{frame}

\section{Bayesian games}

\subsection{Definition}

\begin{frame}{}

\end{frame}

\begin{frame}{}

\end{frame}

\begin{frame}{}

\end{frame}

\begin{frame}{}

\end{frame}


\section{Statistical games}
\subsection{}
\begin{frame}{}

\end{frame}


\begin{frame}{}

\end{frame}

\section{Decisions \& Uncertainty}

\begin{frame}{}

\end{frame}

\section{Summary}

\begin{frame}{}

\end{frame}


\end{document}